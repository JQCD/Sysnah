\documentclass{article}

\usepackage{amsmath}
\usepackage{amssymb}
\usepackage{textcomp}
\usepackage[T1]{fontenc}
\usepackage{graphicx}
\usepackage{listings}
\usepackage[latin1]{inputenc}
\usepackage[a4paper, total={6in, 8.6in}]{geometry}
%\includegraphics[width=10cm]{.png}

\begin{document}
\title{Systemnahes Programmieren }
\author{Michael B�hrer 43279, Stefan Fink 43446, Janet Do 47486}
\maketitle
\tableofcontents

\newpage
\section{Aufgabenstellung}
Es soll die  Funktionsweise eines Scanners sowie die Einordnung innerhalb eines Compilers verstanden werden. Ziel ist es einen funktionieren Compiler zu programmieren. Die Funktionalit�ten eines Compilers liegen grunds�tzlich im Einlesen einer Datei, verarbeiten des eingelesenen Inputs und deren Auswertung. Die Auswertung besteht darin den Code zu zerteilen und in diesem nach vorgegeben Bezeichner, Grundsymbolen, Zahlen etc. auch Tokens genannt, zu filtern. Kommen Symbole vor, die nicht in unserem Vokabular existieren, so werden sie als ung�ltig gewertet. Hierf�r ist der Zustandsautomat verantwortlich. Wurden nun alle Tokens gefunden, so muss ihre Reihenfolge nach einer korrekten Grammatik �berpr�ft werden. Dies �bernimmt der Scanner.
\section{Erl�uterung}
Die Einlese-Komponente wird sp�ter der Buffer. Der Buffer gibt die eingelesenen Zeichen an die Hauptkomponente, den Scanner weiter. Der Scanner hat Zugriff auf den Automaten, Buffer und Symboltabelle. Diese Komponente steuert alles. Der Scanner steuert den Buffer an, wenn er mehr Zeichen braucht, gibt diesen an den Automaten weiter, damit dieser sich immer im korrekten Zustand befindet und �berpr�ft permanent, ob ein Token erkannt wurde. Die Symboltabelle beinhaltet unsere Tokens.
\section{Scanner}
\subsection{Buffer}
	\begin{enumerate}
		\item Konstruktor
		\\ Im Konstruktor wird die Datei mittels des �bergebenen Pfades eingelesen und auf fileDescriptor gespeichert. Ansonsten werden nur die notwendigen Erstinitialisierung, Bufferf�llung und Speicherallozierung vorgenommen.
			\begin{lstlisting}
Buffer::Buffer(const char *path) {
    #ifdef NODIRECT
	if ((this->fileDescriptor = open(path, O_RDONLY)) < 0) {
    #else
    if ((this->fileDescriptor = open(path, O_RDONLY | O_DIRECT)) < 0) {
    #endif
		perror("�ffnen fehlgeschlagen");
		exit(-1);
	}
allocMem(&memptr1, BUFFERSIZE, BUFFERSIZE + sizeof(char));
this->buffer1 = (char*)this->memptr1;
this->buffer1[BUFFERSIZE] = '\0';
this->totalBufferIndex = this->currentBufferIndex = 0;
this->column = this->oldColumn = 1;
this->line = 1;
this->atEof = false;

if (this->readBytes(this->buffer1, BUFFERSIZE) == 0) {
	fprintf(stderr, "Datei ist leer.\n");
	exit(-1);
}
	this->currentBuffer = &this->buffer1;
}
		\end{lstlisting}
			
		\item Diese Methode alloziert den Speicher
		\begin{lstlisting}
inline void Buffer::allocMem(void **mempt, int align, int size) {
	if (posix_memalign(mempt, align, size) != 0) {
		perror("Speicher allozieren fehlgeschlagen");
		exit(-1);
	}
}
		\end{lstlisting}
		
		
\item Destruktor
		\begin{lstlisting}
Buffer::~Buffer(){
	if (this->fileDescriptor > 0) close(this->fileDescriptor);
	free(this->memptr1);
	free(this->memptr2);
}
		\end{lstlisting}

\item Methode berechnet die Anzahl der eingelesenen Bytes mit Hilfe des fileDescriptor und gibt sie zur�ck.
		\begin{lstlisting}
int Buffer::readBytes(char* buf, int count) {
	if (fileDescriptor > 0) {
		int bytesRead = read(this->fileDescriptor, buf, count);
		if (bytesRead < 0) {
			perror("Fehler beim lesen");
			exit(-1);
		}
		if (bytesRead < BUFFERSIZE) buf[bytesRead] = '\0';
		return bytesRead;
	}
	return -1; 
}
		\end{lstlisting}
		
		
\item Holt ein Zeichen aus dem aktuellen Buffer, �berpr�ft davor ob wir am Ende der Zeile angekommen sind. Am Ende
\\ nach dem erfolgreichen Lesen werden die Indizes hochgez�hlt.
		\begin{lstlisting}
char Buffer::getChar() {
    if (this->atEof) return '\0';
	char curr = getCharFromCurrentBuffer();
	
	if (curr == LF) {
		this->oldColumn = this->column;
		this->column = 1;
		this->line++;
	} else this->column++;

    if (curr == '\0') {
        this->atEof = true;
    }

	this->currentBufferIndex++;
	this->totalBufferIndex++;

	return curr;
}
		\end{lstlisting}
		
		
\item Diese Methode wird in der vorherigen, getChar()-Methode aufgerufen. Zu Beginn wird �berpr�ft, ob der 1. BufferIndex �berhaupt schon gesetzt wurde, falls nicht soll dies gemacht werden. Weiterhin weist der negative BufferIndex daraufhin, dass wir weiter "zur�ck" gehen soll im Dokument, wir aber uns bereits am Bufferanfang befinden.
\\Ein m�gliches Szenario: Buffer 1 wurde geladen und gelesen, ein Lexem beginnt am Ende des Buffer 1, Buffer 2 wird bef�llt. Nun wird ein Fehler in Buffer 2 gemerkt, aber da der Anfang von dem Lexem sich in Buffer 1 befindet, muss dieser wieder geladen werden.
\\Ist der BufferIndex bereits gef�llt und gr��er als die vorgegebene BUFFERSIZE, hei�t dies, dass es noch mehr zu lesen gibt, d.h. der n�chste Buffer soll bef�llt werden. 
		\begin{lstlisting}
char Buffer::getCharFromCurrentBuffer() {

	if (this->currentBufferIndex < 0) {
		if (this->currentBuffer == &this->buffer1) {
			this->currentBuffer = &this->buffer2;
		} 
		else if (this->currentBuffer == &this->buffer2) {
		this->currentBuffer = &this->buffer1;
		}
		this->currentBufferIndex = BUFFERSIZE-1;
	} else if (this->currentBufferIndex >= BUFFERSIZE) {
		if (this->currentBuffer == &this->buffer1) {
			if (!this->memptr2) {
			allocMem(&this->memptr2, BUFFERSIZE, BUFFERSIZE 
			+ sizeof(char));
				this->buffer2 = (char*)this->memptr2;
				this->buffer2[BUFFERSIZE] = '\0';
			}
			this->readBytes(this->buffer2, BUFFERSIZE);
			this->currentBuffer = &this->buffer2;
		} 
		else if (this->currentBuffer == &this->buffer2) {
			this->readBytes(this->buffer1, BUFFERSIZE);
			this->currentBuffer = &this->buffer1;
		}
		this->currentBufferIndex = 0;
	}
	return (*this->currentBuffer)[this->currentBufferIndex];
}
		\end{lstlisting}
		
		
\item Diese Methode geht ein Zeichen im Buffer  zur�ck
		\begin{lstlisting}
		void Buffer::ungetChar() {
	if (this->totalBufferIndex == 0) {
		fprintf(stderr, "Kann nicht weiter zur�ckgehen.\n");
		exit(-1);
	}
	this->currentBufferIndex--;
	this->totalBufferIndex--;
	char curr = this->getCharFromCurrentBuffer();
	if (curr == LF) { // dann ans Ende der vorherigen Zeile
		this->column = this->oldColumn;
		this->line--;
	} else this->column--;
		\end{lstlisting}
		
		
\item Getter-Methoden
		\begin{lstlisting}
int Buffer::getBuffersize() {
    return BUFFERSIZE;
}

int Buffer::getCol() {
	return this->column;
}

int Buffer::getLine() {
	return this->line;
}

void Buffer::ungetChar(int count) {
	for (int i = 0; i < count; i++) {
		this->ungetChar();
	}
}

		\end{lstlisting}
		
		
	\end{enumerate}
\subsubsection{Erkl�rung}
\subsection{Automat}
\subsection{Scanner}
\subsection{Symboltabelle}
%\section{Parser}

 

\end{document}