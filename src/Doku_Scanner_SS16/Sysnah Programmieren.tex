\documentclass{article}

\usepackage{amsmath}
\usepackage{amssymb}
\usepackage{textcomp}
\usepackage[T1]{fontenc}
\usepackage{graphicx}
\usepackage{listings}
\usepackage[latin1]{inputenc}
\usepackage[a4paper, total={6in, 8.6in}]{geometry}
%\includegraphics[width=10cm]{.png}

\begin{document}
\title{Systemnahes Programmieren }
\author{Michael B�hrer 43279, Stefan Fink 43446, Janet Do 47486}
\maketitle
\tableofcontents

\newpage
\section{Aufgabenstellung}
Es soll die  Funktionsweise eines Scanners sowie die Einordnung innerhalb eines Compilers verstanden werden. Ziel ist es einen funktionieren Compiler zu programmieren. Die Funktionalit�ten eines Compilers liegen grunds�tzlich im Einlesen einer Datei, verarbeiten des eingelesenen Inputs und deren Auswertung. Die Auswertung besteht darin den Code zu zerteilen und in diesem nach vorgegeben Bezeichner, Grundsymbolen, Zahlen etc. auch Tokens genannt, zu filtern. Kommen Symbole vor, die nicht in unserem Vokabular existieren, so werden sie als ung�ltig gewertet. Hierf�r ist der Zustandsautomat verantwortlich. Wurden nun alle Tokens gefunden, so muss ihre Reihenfolge nach einer korrekten Grammatik �berpr�ft werden. Dies �bernimmt der Scanner.
\section{Erl�uterung}
Die Einlese-Komponente wird sp�ter der Buffer. Der Buffer gibt die eingelesenen Zeichen an die Hauptkomponente, den Scanner weiter. Der Scanner hat Zugriff auf den Automaten, Buffer und Symboltabelle. Diese Komponente steuert alles. Der Scanner steuert den Buffer an, wenn er mehr Zeichen braucht, gibt diesen an den Automaten weiter, damit dieser sich immer im korrekten Zustand befindet und �berpr�ft permanent, ob ein Token erkannt wurde. Die Symboltabelle beinhaltet unsere Tokens.
\section{Scanner}
\subsection{Buffer}
	\begin{enumerate}
		\item Konstruktor
		\\ Im Konstruktor wird die Datei mittels des �bergebenen Pfades eingelesen und auf fileDescriptor gespeichert. Ansonsten werden nur die notwendigen Erstinitialisierung, Bufferf�llung und Speicherallozierung vorgenommen.
			\begin{lstlisting}
Buffer::Buffer(const char *path) {
    #ifdef NODIRECT
	if ((this->fileDescriptor = open(path, O_RDONLY)) < 0) {
    #else
    if ((this->fileDescriptor = open(path, O_RDONLY | O_DIRECT)) < 0) {
    #endif
		perror("�ffnen fehlgeschlagen");
		exit(-1);
	}
allocMem(&memptr1, BUFFERSIZE, BUFFERSIZE + sizeof(char));
this->buffer1 = (char*)this->memptr1;
this->buffer1[BUFFERSIZE] = '\0';
this->totalBufferIndex = this->currentBufferIndex = 0;
this->column = this->oldColumn = 1;
this->line = 1;
this->atEof = false;

if (this->readBytes(this->buffer1, BUFFERSIZE) == 0) {
	fprintf(stderr, "Datei ist leer.\n");
	exit(-1);
}
	this->currentBuffer = &this->buffer1;
}
			\end{lstlisting}
	\end{enumerate}
\subsubsection{Erkl�rung}
\subsection{Automat}
\subsection{Scanner}
\subsection{Symboltabelle}
%\section{Parser}

 

\end{document}